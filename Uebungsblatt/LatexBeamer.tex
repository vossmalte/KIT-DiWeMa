\documentclass{beamer}
\usepackage[utf8]{inputenc}
%%%%%%%%%%%%%%%%%%%%%%%%%%%%%%%%%%%%%%%%%%%%%%%%%%%%%%%%%%%%%%%%%%%%%%%%%%%%%%%%%%%%%%%%%%%%%%%%%%%%%%%%%%%%%%%%%
% Information für die Titelseite der Beamer-Präsentation:
%
\title{Mathematische Textverarbeitung mit \LaTeX}
\subtitle{Eine kurze Einführung im Fachdidaktik-Seminar \\ ``Digitale Werkzeuge für den Mathematikunterricht''}
\author{Ingrid Lenhardt, Judith Schilling}
%\date{WS 18/19}
   
\begin{document}
 
% Erzeugen der Titelseite 
\frame{\titlepage}

%%%%%%%%%%%%%%%%%%%%%%%%%%%%%%%%%%%%%%%%%%%%%%%%%%%%%%%%%%%%%%%%%%%%%%%%%%%%%%%%%%%%%%%%%%%%%%%%%%%%%%%%%%%%%%%%% 
% neue Seite in der Beamer-Präsentation
%
\begin{frame}
\frametitle{Was ist \LaTeX?}  % Überschrift

Softwarepaket zur einfachen Benutzung des Textsatzsystems \TeX 
\begin{itemize} % Auzählung ohne Nummern
 \item \TeX \ wurde von Donald E. Knuth an der Stanford University entwickelt
 \item \LaTeX \ wurde in den 80er Jahren von Leslie Lamport entwickelt
 \item \textbf{kein WYSIWYG} % Wort in Fettdruck 
       (What-you-see-is-what-you-get) wie viele Office-Programme
 \item Autor erstellt Textdateien mit Befehlen zur Formatierung \\ % hier wird Zeilenumbruch erzwungen
       $ \rightarrow $ % Pfeil nach rechts als ein mathematisches Zeichen im Text 
       sauberes Layout, ausgereifter Formelsatz, 
 \item Formeln können im Text eingebunden werden: $\sqrt{2} \approx 1.4$ 
       % Das Dollar-Zeichen markiert Beginn und Ende der Forel im Text 
       oder als abgesetzte Formel gesetzt werden: 
       % hier beginnt die abgesetzte Formel alternativ \begin{equation}
       \[ 
          \int_{-1}^1 \frac{1}{1+x^2} dx = \arctan(1) -\arctan (-1) = \frac{\pi}{2}
       \] 
       % hier endet die abgesetzte Formel alternativ \end{equation}
 \item \LaTeX \ ist ideal für lange Arbeiten, Texte mit vielen Sonderzeichen, Verweisen
\end{itemize}
\end{frame}

%%%%%%%%%%%%%%%%%%%%%%%%%%%%%%%%%%%%%%%%%%%%%%%%%%%%%%%%%%%%%%%%%%%%%%%%%%%%%%%%%%%%%%%%%%%%%%%%%%%%%%%%%%%%%%%%%
%
\begin{frame}
\frametitle{Weitere Vorteile}

\begin{enumerate} % Aufzählung mit Nummern
 \item Es gibt Installationen für die meisten Betriebssysteme
 \item Für den Einstieg gibt es einfach zu bedienende Online-Versionen (z.B. Overleaf, ShareLaTeX) 
 \item Viele Zusatzpakete sind vorhanden (z.B. auch für chemische Strukturformeln)
 \item Ergebnis (i.d.R. PDF-Dokument) ist unabhängig von der Rechnerplattform
 \item Software ist \underline{frei}: % unterstreichen
       \texttt{http://www.latex-project.org/} % Schreibmaschinenschrift
\end{enumerate}

\end{frame}

%%%%%%%%%%%%%%%%%%%%%%%%%%%%%%%%%%%%%%%%%%%%%%%%%%%%%%%%%%%%%%%%%%%%%%%%%%%%%%%%%%%%%%%%%%%%%%%%%%%%%%%%%%%%%%%%%
%
\begin{frame}[fragile]
\frametitle{Erste Schritte in  \LaTeX}
Jede Textdatei startet mit dem Typ des Dokuments, z.B. % in dieser Umgebung werden \LaTeX-Befehle unverändert abgedruckt
{\color{red} % Umschalten auf die Farbe rot - bis die geschweifte Klammer geschlossen wird 
 % in der verbatim-Umgebung werden \LaTeX-Befehle unverändert abgedruckt
\begin{verbatim} 
\documentclass{beamer}, \documentclass{article},  ...
\end{verbatim}
}
\pause 

Im anschließenden Vorspann werden Pakete geladen, Seitenränder (Satzspiegel) vereinbart, eigene Befehle definiert u.Ä., z.B.
{\color{red}
\begin{verbatim} 
 \usepackage{german}          % für Umlaute, Trennung 
 \usepackage{amsmath,amssymb} % für math. Zeichen
 \textwidth170mm              % für Textbreite 
 \newcommand{\R}{\mathbb{R}}  % für reelle Zahlen
\end{verbatim}
}
\pause

Der zu setzende Text steht zwischen

{\color{red}
\verb|\begin{document}|
}
und
{\color{red}
\verb|\end{document}|
}
\end{frame}


%%%%%%%%%%%%%%%%%%%%%%%%%%%%%%%%%%%%%%%%%%%%%%%%%%%%%%%%%%%%%%%%%%%%%%%%%%%%%%%%%%%%%%%%%%%%%%%%%%%%%%%%%%%%%%%%%
%
\begin{frame}[fragile]
\frametitle{Zu guter Letzt}

Die zu dieser Präsentation gehörige Textdatei kann als \textit{Vorlage für eine eigene Präsentation} % Kursivschrift
verwendet werden.
  
Weiter Hilfen zum Arbeiten mit \LaTeX \ (Stand Oktober 2019):
\begin{verbatim} 
https://latex.tugraz.at/latex/tutorial
http://www.nagel-net.de/Latex/DOKU/Latexkurs_Skript.pdf 
http://www.ptep-online.com/ctan/lshort_german.pdf
\end{verbatim} 
\end{frame}
%%%%%%%%%%%%%%%%%%%%%%%%%%%%%%%%%%%%%%%%%%%%%%%%%%%%%%%%%%%%%%%%%%%%%%%%%%%%%%%%%%%%%%%%%%%%%%%%%%%%%%%%%%%%%%%%%
\end{document} % Alles hinter Prozentzeichen oder hinter \end{document} wird als Kommentar behandelt.

Beachte:
Latex erzeugt jede Menge Hilfs-Dateien, die nicht archiviert werden müssen:  *.aux *.log *.nav *.out *.vrb *.snm *.toc
Bilder liegen als eigene Dateien vor. Mit den Bild-Dateien und der *.tex-Datei wird das pdf-Dokument erzeugt.
Manchmal muss mehrmals übersetzt werden, bis alle Verweise stimmen oder das Inhaltsverzeichnis korrekt ist.

