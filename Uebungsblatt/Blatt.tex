\documentclass[a4paper,11pt]{scrreprt} % Die Angaben in eckigen Klammern sind optional (Standardschriftgröße ist 10pt)
%%%%%%%%%%%%%%%%%%%%%%%%%%%%%%%%%%%%%%%%%%%%%%%%%%%%%%%%%%%%%%%%%%%%%%%%%%%%%%%%%%%%%%%%%%%%%%%%
%
% Vorlage für ein Aufgabenblatt im Seminar ``Digitale Werkzeuge im Mathematikunterricht'' am KIT
%
%%%%%%%%%%%%%%%%%%%%%%%%%%%%%%%%%%%%%%%%%%%%%%%%%%%%%%%%%%%%%%%%%%%%%%%%%%%%%%%%%%%%%%%%%%%%%%%%
\usepackage[english,ngerman]{babel} % So ist deutsch die Standardsprache, englisch kann eingeschaltet werden. 
\usepackage[utf8]{inputenc} % Für Umlaute
\usepackage[T1]{fontenc} % Für bessere Silbentrennung
\usepackage{lmodern} % Für eine weniger pixelige Schriftart
\usepackage{graphicx} % Für Bilder (png, pdf, jpg,...)
%
%-------------Mathezeugs--------------
\usepackage{amssymb}
\usepackage{amsthm} % Für mehr Funktionen bei den theorem-Umgebungen
\usepackage{mathtools} % Eine Weiterentwicklung von amsmath
%
%-------Abstände etc.--------------
\usepackage{geometry} % Zur individuellen Anpassung der Seitenränder, z.B.:
\geometry{top=15mm, bottom=15mm, left=20mm, right=15mm}
\usepackage{hyperref}
%
\setlength{\parindent}{0ex} % Kein Einrücken bei einem Absatz
\renewcommand{\arraystretch}{1.5}   % Zeilenbreite bei Tabellen
\pagestyle{empty}                   % keine Seitennummern
%
%----------Kurzbefehle für Mathe-Buchstaben -------------------
\newcommand{\NN}{\mathbb N}
\newcommand{\ZZ}{\mathbb Z}
\newcommand{\QQ}{\mathbb Q}
\newcommand{\RR}{\mathbb R}
%
%------------Theorem-Definitionen----------
% braucht man für Seminararbeiten o.Ä.
\newtheorem{satz}{Satz}[chapter] % Die Nummerierung für Sätze richtet sich so nach Kapiteln
% Das [satz] in den folgenden Definitionen bedeutet, dass alle diese Theoreme fortlaufend mit den Sätzen nummeriert werden (Also "Satz 1.1, Bemerkung 1.2, Satz 1.3" statt "Satz 1.1, Bemerkung 1.1, Satz 1.2" etc.)
\newtheorem{prop}[satz]{Proposition}
\newtheorem{lem}[satz]{Lemma}
\newtheorem{kor}[satz]{Korollar}
\newtheorem{bem}[satz]{Bemerkung}
\theoremstyle{definition} % Die folgenden Umgebungen werden damit nicht kursiv
\newtheorem{defi}[satz]{Definition}
\newtheorem{bsp}[satz]{Beispiel}

%%%%%%%%%%%%%%%%%%%%%%%%%%%%%%%%%%%%%%%%%%%%%%%%%%%%%%%%%%%%%%%%%%%%%%%%%%%%%%%%%%%%%%%%%%%%%%%%
%
\begin{document}

\raisebox{-5mm}{\includegraphics[width=30mm]{kit_logo_de_color.jpg}}
%
\hfill \parbox{22mm}
{ 
Modul \\ Fachdidaktik \\ WS 2019/20 
}

\rule{\textwidth}{1pt}                                   % Linie über die ganze Textbreite
%
\begin{center}
\textbf{
Digitale Werkzeuge für den Mathematik-Unterricht \\[1ex] % Zeilenumbruch mit Abstand
%
% Vortragsnummer und Thema
{
\Large 9. Blatt:  \glqq IMP - Informatik, Mathematik, Physik\grqq } \\[1ex]
% Die Befehle \glqq und \grqq stehen dabei für german left / right double quote. 
%
% Name der Vortragenden und Datum der Erstellung
%
Malte Vo\ss, \today
}
\end{center}
% 
\rule{\textwidth}{1pt}                                 % Linie über die ganze Textbreite                                  
%%%%%%%%%%%%%%%%%%%%%%%%%%%%%%%%%%%%%%%%%%%%%%%%%%%%%%%%%%%%%%%%%%%%%%%%%%%%%%%%%%%%%%%%%%%%%%%%
\vspace*{.5cm} 
 
\textbf{Aufgabe: Graphen}
\begin{enumerate}
    \item[a)] 
        Registriere dich auf der alternativen ILIAS-Plattform unter
        \url{https://scc-ilias-plugins.scc.kit.edu/}. 
        Verwende dazu deinen KIT-Account.

    \item[b)]
        Tritt dem \emph{Kurs Digitale Werkzeuge im Mathematikunterricht} bei.
        Dieser ist im Magazin im Verzeichnis \emph{H5P} oder unter diesem 
        \href{https://scc-ilias-plugins.scc.kit.edu/goto.php?target=crs_5451&client_id=pilot}{Link}
        zu finden.
        Das Kurspasswort lautet \emph{euler}.

    \item[c)]
        Dort liegt die Datei \emph{Graphen - Einführung}. 
        Starte das interaktive Video und beantworte die dort gestellten Fragen.
\end{enumerate}
%
%Hier Aufgabentext einfügen

\vspace{3ex}



%%%%%%%%%%%%%%%%%%%%%%%%%%%%%%%%%%%%%%%%%%%%%%%%%%%%%%%%%%%%%%%%%%%%%%%%%%%%%%%%%%%%%%%%%%%%%%%%

\textbf{Vorbereitung für das Praktikum}
\begin{itemize}
    \item Installiere die Software Jing (\url{https://www.techsmith.com/jing-tool.html}) auf deinem PC.
    \item Bringe die für deinen Vortrag erstellten (Geogebra-) Dateien mit.
\end{itemize}



%%%%%%%%%%%%%%%%%%%%%%%%%%%%%%%%%%%%%%%%%%%%%%%%%%%%%%%%%%%%%%%%%%%%%%%%%%%%%%%%%%%%%%%%%%%%%%%%

\vspace*{\fill} % vertikaler Abstand, der auffüllt, \vfill ginge auch

\begin{flushright}
\textit{Besprechung der Aufgabe am 29. Januar 2020}
\end{flushright}


\end{document} % Alles dahinter wird wie ein Kommentar behandelt!
%
% Formeln im Text beginen und enden mit dem Dollar-Zeichen: $
%
% eine abgesetzte Formel beginnt mit \[ und endet mit \] oder
%
\begin{equation}
 
\end{equation}
%
% Auzählungen mit Nummern:
\begin{enumerate}
 \item[a)]
 \item[b)]
 \item[c)]
\end{enumerate}
%
%Aufzählung ohne Nummern
\begin{itemize}
 \item 
 \item 
 \item
\end{itemize}
%
% Einbinden von Bildern:
\includegraphics[width=  ]{  }
%
% Tabelle mit drei Spalten (link, rechts, zentriert und Trennlinien
% Spalten werden durch & getrennt
\begin{tabular}{|l|c|r|} \hline
 & & \\ \hline
\end{tabular}


